\begin{table}[h]
  \caption{Mixed Evidence for Email
    Mobilization \label{tab:literature}}
  \footnotesize
  \begin{tabular}{p{.3\textwidth}p{.3\textwidth}p{.3\textwidth}}
    \toprule
    \textbf{No Effect} & \textbf{Negative Effect} & \textbf{Positive Effect} \\
    \midrule
    \cite{nickerson2007} & \cite{bennion2011} & \cite{malhotra2012email} \\
    Thirteen field experiments conducted in partnership with politcal
    campaigns find emails have essentially zero effect on voter
    registration or turnout.
                       & University administrators' email
                         encouragement of students to register
                         to vote slightly decreases registration
                         rates.
                                                  & Email messages sent
                                                    from the registrar
                                                    of voters can have
                                                    small, but
                                                    significant
                                                    positive effects
                                                    on turnout.  \\ 
    \bottomrule
  \end{tabular}
  {\raggedright \footnotesize When email has been effective the senders were a trusted, offical
    source \citep{malhotra2012email}. Outside of email mobilization there is
    evidence that some efforts meant to increase turnout can
    inadvertently have the opposite effect \citep[e.g.][]{mccabe2015,
      kousser2007, grose2008, cornwall2012}.\par}
\end{table}

% Stollwerk 2006
